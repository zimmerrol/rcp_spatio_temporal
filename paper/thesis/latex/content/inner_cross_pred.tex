\section{Kreuz-Prädiktion innere Dynamiken}
Bei Messungen der elektrischen Erregung des Herzens können nach aktuellen Stand meistens nur die Erregungen auf der Herzoberfläche gemessen werden. Die Ausbreitungen im Inneren des räumlich ausgedehnten Herzens bleiben somit verborgen. Zudem ist anzunehmen, dass die Gesamtdynamik nicht nur durch die Oberfläche, sondern auch durch die Erregung im Inneren bestimmt und charakterisiert wird. Somit wird die Frage aufgeworfen, ob die innere Erregung des Herzens nur durch die Kenntnis der Oberflächendynamik vorhergesagt werden kann. In diesem Abschnitt soll versucht werden, diese Fragestellung erneut mit den \textsc{ESN}s und den klassischen Methoden zu untersuchen. Dabei wird statt eines dreidimensionalen Systems diese Frage an den zuvor bereits benutzten zweidimensionalen Modellen betrachtet.\\