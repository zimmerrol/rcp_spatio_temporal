\begin{abstract}
  Durch die Verwendung von \textit{Echo State Networks} (\textsc{ESNs}) aus dem Bereich des \textit{Reservoir Computings} konnten in der Vergangenheit Fortschritte bei der Vorhersage zeitlicher Signale erreicht werden. In dieser Arbeit werden sie verwendet, um eine raumzeitliche (chaotische) Dynamik in einem zweidimensionalen System vorherzusagen. Dabei soll eine mögliche Verwendung in der Untersuchung von Herzen betrachtet werden. Dazu wird der Ansatz zuerst auf das \textit{Barkley}- und das \textit{Michell-Schaeffer}-Modell angewendet, welche beide zur Beschreibung von Herzdynamiken genutzt werden, und mit anderen bestehenden Verfahren verglichen. Anschließend wird das komplexere \textit{Bueno-Orovio-Cherry-Fenton}-Modell betrachtet und die vorherigen Erkenntnisse darauf angewendet. Diese Modelle beschreiben ein sogenanntes \textit{erregbares Medium} mit mehreren Systemvariablen.\\
   Im weiteren Verlauf werden drei Fragestellungen betrachtet: Als erstes wird eine Kreuzvorhersage zwischen den Systemvariablen durchgeführt. Darauffolgend wird die Dynamik aus der Kenntnis einer künstlich verschwommenen Messung rekonstruiert. Abschließend werden die Erregungen in ungemessenen Arealen aus der Kenntnis der Randwerte dieser vorhergesagt. Alle Fragestellungen werden sowohl mit den \textsc{ESN}s, als auch mit den klassischen Methoden der \textit{Nächsten-Nachbar}-Vorhersage und der \textit{radialen Basisfunktionen} als Vergleich, bearbeitet. In allen drei Szenarien erreichen die \textsc{ESN}s eine größere Genauigkeit. Sie können die ersten beiden Aufgaben lösen, aber scheitern an der letzten.  

%% Optional: Stichwoerter. Wenn nicht gewuenscht, koennen die beiden
%% folgenden Zeilen geloescht werden
  \bigskip\par
  \textbf{Stichwörter:} Echo State Network, raumzeitliche Dynamik, Chaos, Herzdynamik, Zeitreihenvorhersage, Kreuzvorhersage
\end{abstract}

\clearpage

%% So laesst sich in die andere Sprache umschalten (Englisch bzw. Deutsch)
\begin{otherlanguage}{english}
\begin{abstract}
  Great progress for the prediction of time series has been made in the past by using \textit{Echo State Networks} (\textsc{ESN}s), which are part of the \textit{reservoir computing}. In this thesis they will be used to predict the spatio-temporal (chaotic) dynamics of a two-dimensional system. Thereby, the possible application for studies of the heart shall be considered. Therefore, at first the \textsc{ESN}s are applied on to the \textit{Barkley} and the \textit{Michell-Schaeffer} model, which both can be used to describe the heart's dynamics, and are compared to existing methods. Later, the \textit{Bueno-Orovio-Cherry-Fenton}-model is investigated and the \textsc{ESN} applied another time using the previously obtained insights. These models describe an \textit{excitable medium} with multiple variables.\\
  Three questions will be studied: In the beginning a cross-prediction between the different variables of the systems will be performed. Next, the real dynamics will be predicted by knowing artificial blurred measurements of those. Finally, the excitations of unmeasured regions of the system will be predicted by measuring the boundary values of those. These questions are analyzed using \textsc{ESN}s; the classical methods of the \textit{next neighbour} prediction and the \textit{radial basis functions} are used to compare the performance. While the \textsc{ESN} approach can solve the first two tasks, it fails the last one. But in all three questions it gains a higher accuracy than the two classical approaches.
  \bigskip\par
  \textbf{Keywords:} Echo State Network, spatiotemporal dynmaics, chaos, dynamics of hearts, prediction of time series, cross prediction
\end{abstract}
\end{otherlanguage}