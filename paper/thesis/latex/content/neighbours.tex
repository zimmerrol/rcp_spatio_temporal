\section{Nächste Nachbar Vorhersage}
Das Ziel der \textit{nächsten Nachbar Vorhersage} ist es den funktionalen Zusammenhang $F : X \rightarrow Y$ zu finden, welcher Daten der Menge $X \in \mathbb{R}^n$ auf Elemente aus $Y \in \mathbb{R}^m$ eindeutig abbildet. Hierfür wird angenommen, dass die Funktion $F$ lokal stetig ist. Zudem werden hierfür Daten benötigt, anhand derer der Zusammenhang erlernt werden kann.\\
Zu Beginn werden Paare $(\vec{x},\vec{y}) \in X \times Y$ aus einem \textit{Trainingsdatensatz} gebildet und eine Suchstruktur über die $x$-Werte gebildet. Nun kann diese Struktur genutzt werden, um für ein gegebenes $\vec{x}$ den wahrscheinlichsten Wert $\vec{y}$ zu suchen. Hierfür werden, unter der Annahme der lokalen Stetigkeit, die Datenpunkte $\vec{x}_1, ..., \vec{x}_k$ aus der zuvor angelegten Suchstruktur ausgewählt, welche den geringsten Abstand $d(\vec{x}, \vec{x_i})$ zu $\vec{x}$ besitzen.\\
Diesen $k$ Datenpunkten ist jeweils ein eindeutiger Wert $\vec{y}_i$ zuvor zugeordnet werden. Damit kann nun eine Approximation für den zu $\vec{x}$ gehörigen Wert $\vec{y}$ erstellt werden, indem beispielsweise der Mittelwert der $\vec{y}_i$ genutzt wird.\\

Der Schlüssel in der Bewältigung einer solchen Aufgabe liegt hauptsächlich in der Art und Weise, wie die nächsten Nachbarn gesucht werden. Hierbei werden im Folgenden die beiden Algorithmen $k-d-tree$ und $ball-tree$ ebenso betrachtet wie ein naiver Ansatz. Bei diesem naiven Ansatz (\textit{brute force}) werden die nächsten Nachbarn aus dem unsortierten Trainingsdatensatz durch pures Ausprobieren aller Möglichen Punkte ermittelt.

\subsection{k-d-tree}

\subsection{ball-tree}