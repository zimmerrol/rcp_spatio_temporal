\subsection{Theoretischer Hintergrund}
\label{sc:theory}
Um die mathematischen Eigenschaften beschreiben zu können, sind zuerst zwei Definitionen nötig \cite{yildiz}.

\begin{definition}[Kompatibler Zustand]
Sei $S : X \times U \rightarrow X$ ein \textsc{ESN} mit der Zustandsgleichung $\vec{x}_{n+1} = F \left( \vec{x}_n, \vec{u}_{n+1} \right)$. Eine Folge von Zuständen $(\vec{x}_n)_n$ ist kompatibel mit der Eingangsfolge $(\vec{u}_n)_n$, wenn $\vec{x}_{n+1} = F\left( \vec{x}_n, \vec{u}_{n+1} \right), \forall n \leq 0$ erfüllt ist.
\end{definition}

\begin{definition}[Echo State Eigenschaft (ESP)]
Ein \textsc{ESN} $S : X \times U \rightarrow X$ besitzt die \textit{Echo State Eigenschaft} genau dann wenn eine Nullfolge $(\delta_n)_{n \geq 0}$ existiert, sodass für alle Zustandsfolgen $(\vec{x}_n)_n, (\vec{x}'_n)_n$ die kompatibel mit der Eingangsfolge $(\vec{u}_n)_n$ sind gilt, dass $\forall n \geq 0 ||x_n - x'_n|| < \delta_n$
\end{definition} 
Dies bedeutet, dass nachdem das Netzwerk lang genug betrieben worden ist, der Zustand nicht mehr von dem beliebig gewähltem Anfangszustand abhängt. Diese Eigenschaft ist für notwendig, damit das \textsc{ESN} Vorhersagen treffen kann \cite{jeagerTut2002}.\\

Nun stellt sich die Frage, wann ein Netzwerk diese Eigenschaft besitzt. Es wird schnell klar, dass dies nur durch die Gewichtsmatrix $\mathbf{W}$ bestimmt wird. Betrachtet man die Zustandsgleichung des Netzwerkes, so lässt sich auf Grund des \textit{Banachschen Fixpunktsatzes} erkennen, dass die \textit{ESP} vorliegt, sobald $||\vec{x}_{n+1} - \vec{x}'_{n+1}|| < ||\vec{x}_n - \vec{x}'_n||$ für zwei kompatible Zustände $\vec{x}_n \neq \vec{x}'_n$ erfüllt ist \cite{jaeger2010}.
Hier raus ergibt sich, dass die \textit{ESP} vorliegt, wenn 
\begin{align}
\label{eq:theory_old_requirement}
|1-\alpha(1-\sigma_{max}(\mathbf{W}))| < 1
\end{align}
erfüllt ist, wobei $\sigma_{max}(\mathbf{W})$ der größte Singulärwert ist \cite{jaeger2007}.\\
Weitergehend ist bekannt, dass für Systeme bei denen der Spektralradius $\rho(\mathbf{W}) > 1$ ist diese Eigenschaft nicht vorliegen kann, sofern $\vec{u}_n = 0$ möglich ist \cite{jaeger2007, jaeger2010}.\\

Hier raus ergab sich lange Zeit die falsche Annahme, dass für Systeme mit $\rho(\mathbf{W}) < 1$ die Eigenschaft stets garantiert ist. Wie allerdings gezeigt werden konnte, ist dies nicht der Fall \citep{yildiz}. Stattdessen konnte gezeigt werden, dass eine hinreichende Bedingung durch
\begin{align}
\label{eq:theory_sufficient_requirement}
\rho(\alpha |\mathbf{W}|+(1-\alpha) \mathbf{I}) < 1
\end{align}
gegeben ist - wobei als Betrag der Matrix hier das elementweise Betragsnehmen gemeint ist. Dies ist äquivalent dazu, dass $\rho(|\mathbf{W}|) < 1$ gilt. Diese Bedingung ist weniger einschränkend als \ref{eq:theory_old_requirement} \cite{yildiz}.\\

Darauf basierend kann nun eine Methode nach \cite{yildiz} angegeben werden, um die Gewichtsmatrix $\mathbf{W}$ zu konstruieren:

\singlespacing
\begin{enumerate}
	\item Generiere zufällige Matrix $\mathbf{W}$ mit $\mathbf{|W|} = \mathbf{W}$
	\item Skaliere $\mathbf{W}$, sodass \ref{eq:theory_sufficient_requirement} erfüllt ist.
	\item Wechsel zufällig das Vorzeichen von ungefähr der Hälfte aller Einträge.
\end{enumerate}
\onehalfspacing

Statt dieser Vorschrift wurde zuvor oftmals $\mathbf{W}$ zufälliger generiert und anschließend nur $\rho(\mathbf{W})$ skaliert, was mit unter zu instabilen Systemen geführt hat.