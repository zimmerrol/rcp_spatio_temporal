\section{Fazit}
Wie die Ergebnisse aus \ref{chp:applications} zeigen, bieten \textsc{Echo State Networks} eine gute Möglichkeit zur Vorhersage und Kreuz-Vorhersage von zeitlichen Messreihen. Hierbei konnten teilweise sogar die bisherigen Ergebnisse, welche mit anderen Methoden erzielt worden ist, übertroffen werden. Es hat sich gezeigt, dass für die meisten Anwendungen bereits eine kleine Reservoirgröße $N \leq 2000$ ausreichend ist. Dies hatte zur Folge, dass sowohl der Trainings als auch der Testvorgang sehr schnell und ressourcenschonend durchgeführt werden konnten.\\

Ein Nachteil ist, dass die Modelle über viele Hyperparameter verfügen, welche alle anwengungsspezifisch angepasst werden müssen. In den meisten Fällen konnten diese allerdings relativ schnell so eingestellt werden, dass zufriedenstellende Ergebnisse erzielt werden konnten. Hierbei hat es sich als sehr hilfreich erwiesen erst grobe Parameterbereiche abzutasten um eine Vorauswahl von vielversprechenden Hyperparametern zu finden, bevor diese fein abgetastet werden. 