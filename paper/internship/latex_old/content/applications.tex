\section{Anwendungen}
\label{chp:applications}
Um den Aufwand für die spätere Benutzung zu reduzieren, ist zuerst eine allgemeine Implementation des \textsc{ESN}s geschrieben worden, welche nun anhand zweier exemplarischer Systeme getestet worden ist.

\subsection{Vorhersage eines Rössler-Systems}
Eine erste Anwendung um die Möglichkeiten eines \textsc{ESN} zu demonstrieren besteht in der Vorhersage des Verhaltens eines \textit{Rössler-Systems}
\begin{align}
\label{eq:application_roessler_pde}
\begin{split}
\dot{x} &= -(y+z)\\
\dot{y} &= x + 0.25 \cdot  y\\
\dot{z} &= 0.4 + (x - 8.5)\cdot z.
\end{split}
\end{align}
Dieses System zeigt ein chaotisches Verhalten und stellt somit eine herausfordernde Aufgabe für die zeitliche Vorhersage dar.\\
Hierbei werden zwei verschiedene Aufgaben betrachtet:\textit{(a)} die Vorhersage von $x'(t) \approx x(t+5)$ durch die Kenntnis von $x(t)$ und \textit{(b)} die Kreuz-Vorhersage $y(t)$ durch $x(t)$. Für beide Aufgaben wurde das System für jeweils $6000$ Schritte mit einer Abtastrate $\Delta t = 0.1$ in Analogie zu \cite{parlitz2005} simuliert, wobei die Hälfte der Daten für das Training und der Rest für die Testergebnisse verwendet worden ist.

\subsubsection[Vorhersage $x(t) \rightarrow x(t+5)$]{Vorhersage $\pmb{x(t) \rightarrow x(t+5)}$}
Für diese erste Aufgabe ist durch eine \textsc{GridSearch} ein geeignetes System gefunden werden, dessen Parameter in der Tabelle \ref{tab:application_roessler} dargestellt sind. Dafür wurde ein größerer Bereich des Hyperparameterraumes systematisch abgetastet und die jeweilige Leistung des sich ergebenen \textsc{ESN}s ermittelt, um anschließend die bestmöglichen Parameter zu wählen. Als Maß für diese Optimierung wurde der \textit{Mean Square Error (MSE)}
\begin{align}
MSE = \sum_n \left(x(t+5) - x'(t) \right)^2
\end{align}
zwischen der Vorhersage $x'(t)$ und dem tatsächlichem Wert $x(t+5)$ benutzt.

\begin{table}[H]
	\centering
	\captionsetup{width=0.9\linewidth}
		\begin{tabular}{|c|c|c|}
		\rule[-1ex]{0pt}{3.5ex} Variable & \hspace{4ex} Wert für \textit{(a)} \rule[-1ex]{4ex}{0pt} & \hspace{4ex} Wert für \textit{(b)} \rule[-1ex]{4ex}{0pt}\\ 
		\hline \hline 
		\rule[-1ex]{0pt}{3.5ex} $N$ & $300$ & $500$ \\ 
		\hline 
		\rule[-1ex]{0pt}{3.5ex} $\alpha$ & $0.25$ & $0.20$ \\ 
		\hline 
		\rule[-1ex]{0pt}{3.5ex} $\rho(|\mathbf{W}|)$ & $0.80$ & $0.70$\\ 
		\hline 
		\rule[-1ex]{0pt}{3.5ex} $\nu_{max}$ & $0.01$ & $0.01$\\ 
		\hline 
		\rule[-1ex]{0pt}{3.5ex} $\beta$ & $\num{1e-6}$ & $-$\\ 
		\hline 
	\end{tabular} 
	\caption{Auflistung der Parameter für das jeweilige \textsc{ESN} für (a) und (b).}
	\label{tab:application_roessler}
\end{table}

Der Trainingsvorgang wurde mittels der \textit{Tikhonov Regularisierung} nach Gleichung \ref{eq:tikhonov} durchgeführt. Im Training konnte ein Fehler von $MSE = 0.555 	$ erreicht werden, wobei sich im Testvorgang der Fehler zu $MSE = 0.237$ ergeben hat. Ein graphischer Vergleich zwischen dem Signal $y(t)$ und der Vorhersage $v(n)$ ist in Abbildung \ref{fig:application_roessler_a} zu finden.

\vspace{-3.75ex}
\begin{figure}[H]
    \centering
    \includegraphics[width = 0.9 \textwidth]{figures/roessler_pred50.pdf}
    \caption{Darstellung des Signals $x(t+5.0)$ und der Vorhersage $x'(t) \approx x(t+5.0)$.}
    \label{fig:application_roessler_a}
\end{figure}



\subsubsection[Kreuz-Vorhersage $x(n) \rightarrow y(n)$]{Kreuz-Vorhersage $\pmb{x(n) \rightarrow y(n)}$}
Auch hierfür konnte sehr schnell durch eine \textsc{GridSearch} ein geeignetes System gefunden werden. Als Maß wurde ebenfalls der \textit{MSE} benutzt. Die gefundenen Parameter sind ebenso in der Tabelle \ref{tab:application_roessler} dargestellt.\\
Für das Auslesen wurde eine Linearkombination genutzt, wobei die Lösung $\mathbf{W}_{out}$ mittels der Pseudoinversen nach Gleichung \ref{eq:pseudo_inverse} bestimmt worden ist.\\

Damit wurde für den Trainingsvorgang ein Fehler $MSE_{train} = 0.0025$ und für den Testvorgang von $MSE_{test} = 0.0001$ bestimmt. Die Darstellung der Vorhersage $y'(t)$ und des ermittelten Fehlers sind in den Abbildungen \ref{fig:application_roessler_b1} und \ref{fig:application_roessler_b2} zu finden.

\begin{figure}[H]
    \centering
     %\includegraphics[width = 0.9 \textwidth]{figures/roessler_cross_pred.pdf}
     \includegraphics[width = 0.9 \textwidth]{figures/roessler_cross_err.pdf}
    \caption{Darstellung des Fehlers $y(t)-y'(t)$ zwischen dem Signal $y(t)$ und der Vorhersage $y'(t)$.}
    \label{fig:application_roessler_b1}
\end{figure}
 Im Vergleich zu der vorhandenen Publikation \cite{parlitz2005} ergibt sich ein ähnlicher Fehler. Leider können hierbei nur die Graphen verglichen werden, da keine expliziten Fehler angegeben sind.


\begin{figure}[H]
    \centering
     \includegraphics[width = 0.9 \textwidth]{figures/roessler_cross_pred.pdf}
      \caption{Darstellung des Signals $y(t)$ und der Vorhersage $y'(t)$.}
    \label{fig:application_roessler_b2}
\end{figure}

\clearpage

\subsection{Vorhersage eines Mackey-Glass-Systems}
\label{sec:mackey_glass}
Als nächste Anwendung wird ein \textit{Mackey-Glass-System}
\begin{align}
\label{eq:application_mackey_glass_pde}
\begin{split}
\dot{x}(t) &= \beta \frac{x(t-\tau)}{1+x(t-\tau)^n}-\gamma x(t)\\
\end{split}
\end{align}
betrachtet, wobei $\tau = 17, n=10, \gamma=0.1, \beta = 0.2$ gewählt wurden, um die Ergebnisse abschließend mit den Resultaten von \citep{caraballo2014} vergleichen zu können. Dies ist eine beliebte Aufgabe im Bereich der Vorhersagen chaotischer Zeitreihen.\\
Um die Resultate mit \citep{caraballo2014} vergleichen zu können, wurde für das System in Analogie die Vorhersage $x(t) \rightarrow x(t+84)$ betrachtet. Dabei wurden die Parameter aus Tabelle \ref{tab:application_mackeyglass} genutzt. Mittels der expliziten \textit{Euler-Diskretisierung} 

\begin{align}
x(t+1) = x(t) + \Delta_t \cdot \left(\beta \frac{x(t-\tau)}{1+x(t-\tau)^n}-\gamma x(t)  \right)
\end{align}

sind Werte für $0 \leq t \leq 4000$ simuliert worden, wobei $\Delta_t=\num{1e-2}$ die zeitliche Diskretisierungskonstante ist. Nun sind die ersten $2000$ Werte für den Trainings und die anderen für den Test-Vorgang genutzt worden. Es wurde zur Lösung erneut die \textit{Tikhonov Regularisierung} nach Gleichung \ref{eq:tikhonov} benutzt.

\begin{table}[H]
	\centering
		\begin{tabular}{|c|c|}
		\rule[-1ex]{0pt}{3.5ex} Variable & \hspace{4ex} Wert \rule[-1ex]{4ex}{0pt}\\ 
		\hline \hline 
		\rule[-1ex]{0pt}{3.5ex} $N$ & $1000$ \\ 
		\hline 
		\rule[-1ex]{0pt}{3.5ex} $\alpha$ & $0.20$ \\ 
		\hline 
		\rule[-1ex]{0pt}{3.5ex} $\rho(\mathbf{W})$ & $1.25$ \\ 
		\hline 
		\rule[-1ex]{0pt}{3.5ex} $\nu_{max}$ & $\num{1e-4}$ \\ 
		\hline 
		\rule[-1ex]{0pt}{3.5ex} $\beta$ & $\num{1e-8}$ \\ 
		\hline 
	\end{tabular} 
	\caption{Auflistung der benutzten Parameter des \textsc{ESN}.}
\label{tab:application_mackeyglass}
\end{table}

Interessant ist, dass der Spektralradius größer $1.0$ ist. Dies widerspricht nicht direkt den Aussagen aus Abschnitt \ref{sc:theory}, da diese nur für Signale getroffen werden konnten, welche auch null sein können. Um dies auszuschließen ist ein konstanter \textit{Bias} von $1.0$ als weiteres Eingangssignal genutzt worden.\\
Eine Darstellung der Ergebnisse ist in Abbildung \ref{fig:application_mackeyglass} zu sehen. Der \textit{MSE} wurde für die Trainingsphase als $MSE_{train} = 0.0042$ und für die Testphase als $MSE_{test} = \num{9e-5}$ bestimmt.

\begin{figure}[H]
    \centering
    \includegraphics[width = 0.9 \textwidth]{figures/mackeyglass_pred.pdf}
    \caption{Darstellung des Signals $x(t)$ und der Vorhersage $x(t+84)$.}
    \label{fig:application_mackeyglass}
\end{figure}

Dies ist eine starke Verbesserung im Vergleich zu dem besten Ergebnis aus \citep{caraballo2014} mit $MSE = 0.0014$, wobei anzumerken ist, dass dort eine kleinere Trainings- und Testphase gewählt worden ist.